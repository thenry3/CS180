\documentclass[12pt]{article}
\usepackage[left=0.5in,right=0.5in,top=0.5in,left=0.5in]{geometry}
\usepackage{algpseudocode}
\usepackage{algorithm}

\title{\textsc{Homework 2}}
\author{Henry Trinh}

\begin{document}
\maketitle

\newpage % PROBLEM 1 %
\section*{Problem 1}
We can prove that in a binary tree, the number of nodes with two 
children is exactly one less than the number of leaves using induction.
\subsection*{Proof}
For our base statement, we look at a binary tree with only one node, the root node.
In this example, the statement is true because there is only one leaf and no full nodes.
\newline
\newline
For any positive integer $N$, suppose that the statement is true for all binary trees
with $N$ nodes. Given a binary tree $B$ with $N+1$ nodes, say we have a leaf $L$ with
parent $L^\prime$ and remove the leaf $L$, giving us the binary tree $B^\prime$ with 
$N$ nodes. We then have two cases in which $L^\prime$ has two children or one child in 
$B$:
\begin{enumerate}
    \item In $B^\prime$, if $L^\prime$ is a leaf, then $B^\prime$ has the same number of leaves
    as $B$ since $L$ is removed from the tree and $L^\prime$ is converted from a node to 
    a leaf. The number of nodes with two children stay the same between $B$ and $B^\prime$ 
    since $L^\prime$ only had one child in $B$ and did not count as a full node in $B$.

    \item If $L^\prime$ has one child in $B^\prime$, then $B^\prime$ has one less leaf and one less 
    node with two children than $B$ since $L$ was removed and $L^\prime$ is no longer counted as
    a full node with two children.
\end{enumerate}
In both of these cases the difference of the number of nodes with two children and number
of leaves stays the same between $B^\prime$ and $B$, where the difference is $1$. By 
induction, the above statement is true for any positive integer $N$.

\newpage
\section*{Problem 2}

\begin{algorithm}
\caption{Shortest path counts for each node}
\begin{algorithmic}[1]
    \State $dist \gets $ initalize array of length $n$ with value of $Null$
    \State $paths \gets $ initialize array of length $n$ with $0$
    \State $dist[src] = 0$
    \State $paths[src] = 1$
    \State initalize empty Queue $Q$
    \State push $src$ onto $Q$
    \While {$Q$ is not empty}
        \State $u \gets $ dequeue $Q$
        \ForAll {$v$ adjacent to $u$}
            \If {$dist[v]==Null$}
                \State $dist[v] \gets dist[u]+1$
                \State $paths[v]==paths[u]$
                \State push $v$ onto $Q$
            \Else
                \If {$dist[v]==dist[u] + 1$}
                    \State $paths[v] += paths[u]$
                \EndIf
            \EndIf
        \EndFor 
    \EndWhile
    \State return $paths$
\end{algorithmic}
\end{algorithm}
\subsection*{Explanation}
This algorithm is correct because it visits every node by traversing every edge using BFS while 
keeping track of the distance it takes to get to the node from the source node as well as the number
of shortest paths for every node. Since this is an unweighted graph, when a previous node is revisited,
the proposed distance will always be greater than or equal to the recorded shortest distance. Because of this property
of BFS to traverse by "levels", we only need to check if the node is unvisited or if the proposed distance
is equal to the recorded shortest distance. If the node is unvisited, it inherits the number of paths as from
the previous node because each of those paths from the previous node only has one direction to go. If the
proposed distance is the same as the reocrded distance, then the number of paths from the previous node is 
added to the current number of paths since it adds a new direction and alternative routes. We can continue
this pattern until there are no more unvisited nodes.

\newpage
\section*{Problem 3}

\subsection*{Part A}
\vspace{-5mm}
\begin{algorithm}
\caption{Find diameter of G}
\begin{algorithmic}[1]
    \State $shortest \gets$ matrix with $n$ rows and columns initialized with all values as infinity
    \ForAll {vertex $v$}
        \State $shortest[v][v] \gets 0$
        \State initialize queue $Q$
        \State push $v$ onto $Q$
        \State initalize set $S$
        \While {$Q$ is not empty}
            \State $u \gets $ dequeue $Q$
            \ForAll {$w$ adjacent to $u$}
                \If {$w$ in $S$}
                    \State put $w$ in $S$
                    \State $shortest[v][w] \gets shortest[v][u] + 1$
                    \State push $w$ in $Q$
                \EndIf
            \EndFor
        \EndWhile
    \EndFor
    \State return $max(shortest)$
\end{algorithmic}
\end{algorithm}

\newpage
\section*{Problem 3}

\subsection*{Part B}
\vspace{-5mm}
\begin{algorithm}
\caption{Find diameter of T}
\begin{algorithmic}[1]
    \Function{TreeDiameter}{root}
        \If {$root$ has no children}
            \State return $0$
        \EndIf
        \State $max1 \gets 0$
        \State $max2 \gets 0$
        \ForAll {child $c$ of node $root$}
            \State $dist \gets TreeDiameter(c) + 1$
            \If {$dist > max1$ or $dist > max2$}
                \If {$max1 > max2$}
                    \State $max2 \gets dist$
                \Else
                    \State $max1 \gets dist$
                \EndIf
            \EndIf
        \EndFor
        \State return $max1 + max2$
    \EndFunction
\end{algorithmic}
\end{algorithm}

\newpage
\section*{Problem 4}
\subsection*{Part A}
\begin{algorithm}
\caption{Test if DAG is semi-connected}
\begin{algorithmic}[1]
    \State $topArr \gets $ array of vertexes returned from topological sort on $G$
    \For {$i$ in $topArr$}
        \If {$(topArr[i],topArr[i+1]) \notin E$}
            \State return $False$
        \EndIf
    \EndFor
    \State return $True$
\end{algorithmic}
\end{algorithm}

\subsection*{Part B}
We have a graph $G^\prime$ constructed from Kosaraju's algorithm
performed on graph $G$ such that $G^\prime = (V^\prime, E^\prime)$ where $V^\prime$
has $K$ nodes $v_1^\prime,...,v_k^\prime$ that are each a Strongly Connected Component (SCC)
that is maximal, and $(v_i^\prime, v_j^\prime) \in E^\prime$ if and only if there exists
a edge connecting a pair of nodes in $v_i^\prime$ and $v_j^\prime$.
\newline
\newline
To prove that $G^\prime$ is a Directed Acyclic Graph, we can prove by contradiction.
Suppose that $G^\prime$ is a directed graph with a cycle between two nodes $v_i^\prime$
and $v_k^\prime$, which are SCCs. This implies that there exists a path of nodes in $G$ 
and $v_k^\prime$ such that if the nodes are added to $v_i^\prime$, $v_i^\prime$ would 
remain a Strongly Connected Component. This is a contradiction to the fact that all nodes
in $V^\prime$ are maximal SCCs. This means that graph $G^\prime$ must have no cycles, 
therefore making it a DAG.

\subsection*{Part C}
\begin{algorithm}
\caption{Test if directed graph G is semi-connected}
\begin{algorithmic}
    \State $S \gets $ set of SCCs of $G^\prime$ returned from Kosajuru's algorithm performed on $G$
    \State $topArr \gets $ array of SCCs returned from topological sort on the graph made by $S$
    \For {$i$ in $topArr$}
        \If {$(topArr[i],topArr[i+1]) \notin E^\prime$}
            \State return $False$
        \EndIf
    \EndFor
    \State return $True$
\end{algorithmic}
\end{algorithm}


\end{document}

