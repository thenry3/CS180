\documentclass[12pt]{article}
\usepackage[top=0.5in, left=0.5in, right=0.5in, bottom=00.5in]{geometry}
\usepackage{algpseudocode}
\usepackage{algorithm}

\title{\textsc{Homework 4}}
\author{Henry Trinh}

\begin{document}
\maketitle

\newpage
\section*{Problem 1}
\subsection*{Part A}
Given $n=5$ days, $s=\{50,\ 10,\ 5,\ 3,\ 2\}$, and $x=\{100,\ 100,\ 100,\ 100,\ 100\}$,
we have an example where the optimal solution includes at least $2$ reboots. The optimal
solution is to reset in days 2 and 4. This gives us a solution of $50 + 0 + 50 + 0 + 50 = 150$
terabytes processed.

\subsection*{Part B}
\begin{algorithm}
\caption{Return max amount of terabytes that can be processed}
\begin{algorithmic}[1]
\Function{TerabytesProcessed}{$x$, $s$}
    \State $n \gets $ length of $x$
    \State $arr \gets $ initalized array of $0$'s of length $n$
    \State $ptr \gets 0$
    \For {$i$ in the range of $0...n-1$}
        \State $base1 \gets arr[i - 2]$ if $i > 1$ otherwise $0$
        \State $base2 \gets arr[i - 1]$ if $i > 0$ otherwise $0$
        \State
        \State $temp1 \gets base1 + min(x[i],\ s[0])$
        \State $temp2 \gets base2 + min(x[i],\ s[ptr])$ 
        \State
        \State $arr[i] \gets max(temp1,\ temp2)$
        \If {$arr[i]==temp1$}
            \State $ptr \gets 0$
        \EndIf
        \State $ptr \gets ptr + 1$
    \EndFor
    \State return last element in $arr$
\EndFunction
\end{algorithmic}
\end{algorithm}
This algorithm works using the concept of dynamic programming. We can solve this by using a bottom
up approach to build the maximum value up until the last day. We can classify each day as a subproblem
where there are two options. The first case is where we restart the computer the day before, or continue
processing the data from the day before. We can calculate and see which of the two options gives the maximal 
value for the current day. This method can be continued for the following days and takes into account of 
whether or not it's better to reset on a day based on the day after.
\newline
\newline
This algorithm has a linear time complexity of $O(n)$ since the algorithm iterates through the number of $n$ days.

\newpage
\section*{Problem 2}
\begin{algorithm}
\caption{Return minimum number of characters required to make string a palindrome}
\begin{algorithmic}[1]
    \State $s \gets $ given string
    \State $n \gets $ length of $s$
    \State $dp \gets $ 2d array of size $n$ x $n$ initalized with null values
    \ForAll {$i$ in the range from $0...n-1$}
        \State $dp[i][i] \gets 0$
    \EndFor
    \ForAll {$i$ in range from $n-2...0$}
        \ForAll {$j$ in the range from $i+1..n-1$}
            \If {$s[i] == s[j]$}
                \If {$j - i == 1$}
                    \State $dp[i][j] \gets 0$
                \Else
                    \State $dp[i][j] \gets dp[i + 1][j - 1]$
                \EndIf
            \Else
                \State $dp[i][j] \gets min(dp[i+1][j],\ dp[i][j-1]) + 1$
            \EndIf
        \EndFor
    \EndFor
    \State return $dp[0][n - 1]$
\end{algorithmic}
\end{algorithm}
This algorithm uses a bottom up approach to find the number of insertions needed to make a string
a palindrome. We can utilize a two pointer approach to mark the current substring/subproblem we are 
examining of the given string. We can start with the base cases such that a sole subproblem/substring of 
length 1 takes 0 insertions since a sole letter is a palindrome itself. Our second base case is if the length
of the subproblem is of length 2, in which case if both letters are equal, it takes 0 insertions, and otherwise
it will take 1 insertion. For any given substring, we have two cases. If the first and last letters are equal, 
then the number of insertions is equal to the number of insertions of the substring without those two letters 
since the two letters being equal guarantees that they do not affect the number of insertions. Otherwise, 
in the second case where they are not equal, the number of insertions will be the minimum of either 
the substring of (left + 1, right) or (left, right - 1) and adding 1 to the minimum since the character left out
needs to be accounted for. Through this, we can continue to fill out a 2d array of size $n$ x $n$, which stores
the answers and builds up from the base answers in the array.
\newline

This algorithm has a time complexity of $O(n^2)$ since this algorithm fills out an array of 
size $n^2$. The initial loop to fill out the base cases only take a time complexity of $O(n)$, so it does not affect 
the final time complexity.

\newpage
\section*{Problem 3}
\begin{algorithm}
\caption{find maximum cardinality set in tree $T$}
\begin{algorithmic}
\Function{findMaxCardSet}{tree $T$}
    \State $dp \gets $ initialize hashmap from node to integer
    \State $root \gets $ root of tree $T$
\EndFunction
\State
\Function{helper}{vertex $u$, hashmap $dp$}
    \If {$u \in dp$}
        \State return $dp[u]$
    \EndIf
    \If {$u$ has no children}
        \State $dp[u] = 1$
        \State return $1$
    \EndIf
    \State
    \State $temp1 \gets 0$
    \State $temp2 \gets 0$
    \ForAll{children $v$ of $u$}
        \State $temp1 += helper(v,\ dp)$
        \ForAll {children $w$ of $v$}
            \State $temp2 += helper(w,\ dp)$
        \EndFor
    \EndFor
    \State $currMax \gets max(temp1,\ temp2 + 1)$
    \State $dp[u] \gets currMax$
    \State return $currMax$
\EndFunction
\end{algorithmic}
\end{algorithm}

This algorithm takes a top down dynamic programming approach by calculating the maximal cardinal 
set in each subtree and saving the result in a hash map to prevent recalculation of already calculated 
subtrees. At the smallest subproblem/subtree is a single vertex in the tree without any children. 
In this case, the size of the maximum cardinality set in that subtree is 1 since the vertex itself is 
a maximal set. We can divide every other subtree with children into two cases. The first case is where the 
size of the maximal set is the sum of the size of all the maximal sets of its children. The second case is 
the same, except with its grandchildren, and added 1 to the number to represent the additional vertex, 
which is the root of the current subtree. The size of the max cardinality set of the current subtree would 
be the maximum of these two cases. This pattern can be repeated for every single vertex in the tree and be 
used to finally determine the size of the maximum cardinality set in the tree.
\newline

The runtime of this algorithm is $O(V)$ since this algorithm visits each node once and does not recalculate
the algorithm for each vertex that already has been already calculated.

\newpage
\section*{Problem 4}
We can prove that the vertex cover problem $<=_p$ hitting set problem by giving a polynomial time reduction from
the vertex cover problem. Given a graph $G(V,\ E)$ of the vertex cover problem, we can construct a hitting set problem
by constructing a set $B = {u, v}$ for every edge $(u,\ v) \in E$ and the value of $k$ that remains equal for both problems.
With this in mind, we can reach the conclusion that $G$ has a vertex cover of size $k$ if there is a hitting set $H$ of size at most $k$ of sets 
$B_i$.
\newline
\newline
A further proof is as follows. If there is a graph $G$ with vertex cover $VC$, then for every edge $(u,\ v)$, either 
$u \in VC$ or $v \in VC$. This means in a hitting set situation $\{u,\ v\} \cap VC \neq \emptyset$, which means that 
$H$ is a hitting set of size $k$. This also works conversely, where for a given hitting set $H$, every edge is covered
by the hitting set $H$, and $H$ is a hitting set of graph $G$ of size $k$.

\newpage
\section*{Problem 5}
Given a graph $G(V,\ E)$, we can proceed to prove that the 3-colorable problem $<=_p$ the 4-colorable problem. Given a 
second graph identical to $G$, $G^\prime(V^\prime,\ E^\prime)$, we can introduce a new vertex $u$ such that $V \cup \{u\} = V^\prime$.
We can also assume that $u$ has an edge connected to every existing vertex $v$ in $G$ such that 
$E^\prime = E \cup \{(u,\ v) \forall v \in V\}$. 
\newline
\newline
If there is a 3-coloring $col = \{0,\ 1,\ 2\}$ of $G$, we can also define $col^\prime = \{0,\ 1,\ 2,\ 3\}$ as a 4-coloring.
We can define their relationship such that $\forall v \in V$, $col^\prime(v) = col(v)$ and $col^\prime(u) = 3$. We can then associated
$col^\prime$ as the 4-coloring of $G^\prime$. This shows us that for any graph $G$ that is 3-colorable, $G^\prime$ can also be 4-colorable.
We can also show the converse of this, where given the same graphs, since the color 3 is only assigned to $u$ and it is 
connected to all vertices in $V$, only $u$ has the fourth color. This shows that the 3-colorable $col$ is a 3-colorable of $G$. This also shows
that if $G^\prime$ is 4-colorable, then $G$ is colorable by 4 colors.

\end{document}